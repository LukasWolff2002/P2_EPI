\section{Lean Construction}

\begin{itemize}
    \item Enfoque tradicional versus enfoque lean en la construcción.
    \item Modelos de flujo y de transformación.
    \item Principios del Lean Construction.
    \item Muda, Mura y Muri.
    \item Tipos de desperdicio en la construcción.
\end{itemize}

\subsection*{Comparación: Enfoque Tradicional vs Enfoque Lean en la Construcción}

\begin{table}[h]
    \centering
    \small
    \begin{tabular}{|p{6.5cm}|p{6.5cm}|}
        \hline
        \textbf{Enfoque Tradicional} & \textbf{Enfoque Lean} \\
        \hline
        Sigue modelo de transformación & Sigue el modelo de flujos \\
        \hline
        Primero se diseña el producto y después se diseñan los procesos & Los productos y procesos son diseñados conjuntamente \\
        \hline
        No todas las etapas del ciclo de vida del producto son consideradas durante el diseño & Todas las etapas del ciclo de vida del producto son consideradas durante el diseño \\
        \hline
        Las actividades se llevan a cabo tan pronto como sea posible & Las actividades se llevan a cabo al último momento responsable \\
        \hline
        Se eligen los subcontratistas en función del costo & Se eligen los subcontratistas debido a su capacidad de colaboración \\
        \hline
    \end{tabular}
    \caption{Comparación entre el enfoque tradicional y el enfoque Lean en la construcción}
\end{table}

\subsection*{Principios de Lean Construction}

\begin{enumerate}
    \item Reducción o eliminación de actividades que no agregan valor.
    \item Incremento del valor del producto considerando los requerimientos del cliente.
    \item Reducción de la variabilidad.
    \item Reducción del tiempo del ciclo.
    \item Simplificación de pasos, partes o enlaces.
    \item Incremento de la flexibilidad de la producción.
    \item Aumento de la transparencia del proceso.
    \item Enfoque del control al proceso completo.
    \item Mejoramiento continuo del proceso.
    \item Equilibrio entre el mejoramiento de flujos y el mejoramiento de la conversión.
    \item Benchmarking.
\end{enumerate}

\subsection*{Distribución del Trabajo}

\begin{itemize}
    \item \textbf{Trabajo Productivo (TP)}: Actividades que \textbf{agregan valor}.
    
    \item \textbf{Trabajo Contributorio (TC)}: Actividades que \textbf{sirven de apoyo y son necesarias}, pero \textbf{no agregan valor directamente}.
    
    \item \textbf{Trabajo No Contributorio o No Productivo (TNC)}: Actividades \textbf{innecesarias}, que tienen un \textbf{costo asociado} y \textbf{no agregan valor}.
\end{itemize}

\subsection*{Desperdicios}

\noindent Los \textbf{desperdicios} son cualquier elemento de producción, procesamiento o distribución que \textbf{no agrega valor} al producto final.

\vspace{0.3cm}

\noindent Los desperdicios \textbf{únicamente añaden costo y tiempo} a un proceso, sin contribuir al resultado esperado.

\subsection*{Desperdicios (Mudas)}

\begin{itemize}
    \item Transportación
    \item Defectos
    \item Sobreproducción
    \item Sobre-procesamiento
    \item Talento no utilizado
    \item Inventario
    \item Movimientos 
    \item Espera
\end{itemize}

\subsubsection*{Transportación}
Movimientos innecesarios de recursos (personas, equipos o materiales) desde un proceso a otro
Posibles causas:
\begin{itemize}
    \item Condiciones de terreno o proyecto
    \item Materiales no almacenados en la ubicación correcta
    \item Materiales recibidos con mucha anticipación
    \item Falta de plan logístico
\end{itemize}

\subsubsection*{Defectos}
Actividad que requiere retrabajo por errores u omisiones
Posibles causas:
\begin{itemize}
    \item Estándares de calidad mal definidos
    \item No seguir los procedimientos
    \item Falta de información
    \item Objetios no comunicados oportunamente   
\end{itemize}


\subsubsection*{Sobreproducción}
Ejecutar una actividad antes de que sea realmente necesarias
Posibles causas:
\begin{itemize}
    \item Falta de comunicación
    \item Adelantarse al Programa
    \item Deseo de ahorrar en mano de obra
    \item Deseo de moverse a otras actividades
\end{itemize}

\subsubsection*{Sobre-procesamiento}
Movimientos innecesarios en una obra de personas, equipos o materiales desde un proceso a otro. Esto puede incluir trabajo administrativo, así como actividades físicas.
Posibles causas:
\begin{itemize}
    \item Baja coordinación de equipos
    \item No conocer las especificaciones
    \item Contar con procesos poco optimizados
    \item Usar equipos excesivamente sofisticados
\end{itemize}

\subsubsection*{Esperas}
Interrupciones del trabajo o tiempo de inactividad o por falta de insumos
Posibles causas:
\begin{itemize}
    \item No saber qué es lo que se tiene que hacer
    \item Problemas no resueltos para un trabajo continuo
    \item Falta de equipamiento, herramientas o materiales
\end{itemize}

\subsubsection*{Movimientos}
Desplazamioentos innecesario de personal o maquinaria durante su trabajo, ya que dejan de producir durante su traslado
Posibles causas:
\begin{itemize}
    \item Falta de planificación
    \item Falta de organización
    \item No estudiar y aplicar los métodos más eficientes
    \item Problemas no resueltos para un trabajo continuo
\end{itemize}

\subsubsection*{Inventario}
Cantidad de materiales que va por sobre la necesidad inmediata. Además, de materiales puede incluir trabajo en proceso y productos terminados.
Posibles causas:
\begin{itemize}
    \item Problemas de planificación
    \item Poca confianza en los equipos de trabajo o equipo poco eficientes
    \item Falta de coordinación de equipos en obra
\end{itemize}


\subsubsection*{Talento no utilizado}
Desaprovechar el potencial de las personas en la organización
Posibles causas:
\begin{itemize}
    \item Falta de un método estándar para capturar ideas
    \item Cada nuevo proyecto es un nuevo inicio
    \item No incluir toda la cadena de producción en la planificación o soluciones
\end{itemize}

Bajando el nivel de desperdicios, exponemos los problemas reales: 
\begin{itemize}
    \item Tiempo excedente (buffers)
    \item dinero y materiales (inventario)
\end{itemize}

Meta:
\begin{itemize}
    \item Identificar Desperdicios
    \item Remover Desperdicios
    \item Revelar problemas
    \item Resolver los problemas
\end{itemize}

\textbf{Productos más rápidos, económicos y mejores}

\subsection{Cultura Lean}
\begin{itemize}
    \item Mejoramiento continuo.
    \item Respeto por las personas.
    \item Enfocarse en que todos piensen como empresa
    \item Competente
    \item Flexible
    \item Empoderados
    \item Motivados
\end{itemize}

Enfoque de arriba hacia abajo, donde pocos son recompensados $\rightarrow$ Ambiente de empoderamietno, con una fuerza laboral plenamente educada que disfrute de mayores incentivos

"Las personas no se resisten al Lean. Las personas se resisten a las formas en que perciben como Lean afectará sus vidas"

\begin{table}[h]
\centering
\begin{minipage}{0.45\textwidth}
\raggedright
\textbf{La gente resiste desafíos a su:}
\begin{itemize}
    \item Visión del mundo
    \item Narrativa
    \item Creencias
    \item Ego y autoimagen
\end{itemize}
\end{minipage}
\hfill
\begin{minipage}{0.45\textwidth}
\raggedright
\textbf{La gente ofrece resistencia por miedo a:}
\begin{itemize}
    \item Dolor al cambio
    \item Pérdida
    \item Seguridad
    \item Lo desconocido
\end{itemize}
\end{minipage}
\end{table}



\begin{table}[h]
\centering
\begin{minipage}{0.45\textwidth}
\raggedright
\textbf{Todos somos solucionadires de problemas:}
\begin{itemize}
    \item Aprende de los experimetos de otros
    \item Cuestionar los sesgos y supuestos
    \item Conoce sobre buenos argumentos y malos argumentos
    \item Planifica a largo plazo
    \item Define lo que debe y no se debe hacer
\end{itemize}
\end{minipage}
\hfill
\begin{minipage}{0.45\textwidth}
\raggedright
\textbf{Aprende de los experimentos de otros:}
\begin{itemize}
    \item El contexto y las condiciones son muy importantes
    \item No vayas directo a buscar respuestas o conclusioes de otros
    \item Resuelve tu problema específico
    \item Como un paciente responsable, no tomas medicina de otros
\end{itemize}
\end{minipage}
\end{table}


\newpage

\subsection*{Tecnología: Lean Construction}

\begin{multicols}{2}
    \begin{itemize}
        \item 5 porqués
        \item 6S
        \item 8 desperdicios
        \item Diagramas de causa y efecto
        \item DMAIC
        \item Flujo de valor
        \item Gemba
        \item Genchi genbutsu
        \item Gráficos de control
        \item Gráficos de Pareto
        \item Histograma
        \item Mapeo del flujo de valor (VSMA)
        \item Pensamiento y herramienta A3
        \item Planificar-hacer-estudiar-actuar (PDSA)
        \item Pull
        \item Trabajo equilibrado
        \item Trabajo estándar
        \item LPS
        \item VDC, TVD, entre otros
    \end{itemize}
\end{multicols}




