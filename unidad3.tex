\section{Target Value Design (TVD)}

Metodo que se ideo con el objetivo de conducir el diseño y construccion de un proyecto
hacia el \textbf{logro de los objetivos}

Se busca hacer una adatacion del \textbf{target costing} 

Las diferencias entre el diseño convencional y el TVD son:

\begin{table}[h!]
\centering
\begin{tabular}{|p{6.5cm}|p{6.5cm}|}
\hline
\textbf{TVD} & \textbf{Práctica tradicional} \\
\hline
Ciclo de diseño – estimación – rediseño & Ciclo de diseño – estimación – retrabajo \\
\hline
Se elabora primero un estimado y entonces el diseño es realizado alineándose con el estimado. & El diseño multidisciplinario es dibujado y entonces se hace el estimado en función de él. \\
\hline
El diseño se basa solo en lo que se puede construir. & Es necesaria una evaluación del diseño respecto a constructabilidad. \\
\hline
Todos los diseñadores se involucran desde el diseño inicial. & El arquitecto diseña, entonces otros involucrados basan su trabajo en el proyecto de arquitecto. \\
\hline
El costo objetivo nunca debe ser excedido. & El costo del proyecto excede lo que el cliente puede pagar por él. \\
\hline
\end{tabular}
\caption{Comparación entre TVD y práctica tradicional}
\end{table}

Ahora, para poder llevar a cabo el TVD, se deben cumplir los siguientes aspectos:

\begin{itemize}
    \item Integracion Temprana
    \item Compartir Riesgos y Recomensas
    \item Tener Objetivos Alineados
    \item Diseño Concurrente, compartir el conocimiento sobre costos y experiencias.
\end{itemize}

\begin{figure}[H]
\centering
\includegraphics[width=0.8\textwidth]{IMAGENES/TVD.png}
\caption{Proceso de TVD}
\label{fig:tvd}
\end{figure}

Los pasos a ejecutar son:

\begin{itemize}
    \item Definir el costo objetivo para el proyecto
    \item Trabajar estructuradamente
    \item Colaboracion
    \item Set-Based Design
    \item Collocation
\end{itemize}

Ahora bien, algunas buenas practicas a seguir dentro de TVD son:

\begin{itemize}
    \item Comprometerse con el cliente para \textbf{establecer valor objetivo}
    \item Guiar los esfuerzos de diseño hacia el \textbf{aprendizaje y la innovación}
    \item \textbf{Diseñar} en relación con el \textbf{presupuesto y el valor objetivo} del cliente
    \item Planear \textbf{colaborativamente}
    \item Diseñar simultáneamente el \textbf{producto y el proceso}
    \item Diseñar y planear con base en el \textbf{cliente} que utilizará el producto
    \item Trabajar en \textbf{pequeños grupos} multidisciplinarios
    \item Trabajar en un “Big room”. \textbf{Co-location}
    \item Realizar \textbf{retrospectivas} a lo largo del proceso
\end{itemize}

\subsection{Como trabajar en TVD}

De esta forma, se ontiene lo siguiente:


\noindent En lugar de diseñar aisladamente y posteriormente reunirse para revisiones y decisiones de grupo, \textbf{colaboración} 
\[
\Rightarrow \text{Trabajar en equipo para definir los inconvenientes y decisiones y luego diseñar conforme a esas decisiones}
\]

\noindent En lugar de evaluar la constructabilidad del diseño, \textbf{estructura de trabajo}
\[
\Rightarrow \text{Diseñar lo que es construible}
\]

\noindent En lugar de trabajar solos en cuartos separados, \textbf{co-locación}
\[
\Rightarrow \text{Trabajar en parejas o en grupos más grandes, cara a cara}
\]

\noindent En lugar de estimar con base en un diseño detallado, \textbf{costo objetivo}
\[
\Rightarrow \text{Diseñar con base en un estimado detallado}
\]

\noindent En lugar de tomar decisiones que reduzcan las posibilidades para proceder con el diseño, \textbf{Set Based Design}
\[
\Rightarrow \text{Mantener conjuntos de decisiones abiertos a lo largo del proceso de diseño}
\]

\subsection{Beneficios del TVD}

\begin{itemize}
    \item Posicionamiento hasta un 15\% por debajo del precio de mercado
    \item Reducción de costos sin comprometer la calidad, el cronograma o el alcance del proyecto
    \item Reducción de tiempos de ejecución
    \item Buenas relaciones y ausencia de conflictos
    \item Ausencia de reclamos
\end{itemize}

\section{Integrated Project Delivery (IPD)}

\textbf{Claridad} implica:

\begin{itemize}
    \item Conocer cuales son las reglas
    \item Conocer que es ganar
\end{itemize}

\textbf{Alineamineto} implica:

\begin{itemize}
    \item Mismas reglas
    \item Mismas ganancias
\end{itemize}

El IPD es la metodologia que \textbf{Alinea Colaborativamente} a las personas, a los sistemas y los procesos del negocio, para \textbf{Aprovechar}
los talentos de los participantes, asi pueden \textbf{optimizar el proyecto} reduciendo el valor 

De esta forma, el IPD se ve como:

\begin{figure}[H]
\centering
\includegraphics[width=0.8\textwidth]{IMAGENES/IPD.png}
\caption{Proceso de IPD}
\label{fig:ipd}
\end{figure}

De esta forma:

\begin{itemize}
    \item Se basa principalmente en la colaboración y la confianza.
    \item Genera buenos resultados siempre y cuando las personas se respeten mutuamente.
    \item Se centran en obtener buenos resultados para el proyecto y no se desvíen en lograr metas individuales.
\end{itemize}

Donde sus principios son:

\begin{itemize}
    \item Respeto mutuo y confianza
    \item Beneficio mutuo y recompensa
    \item Innovación colaborativa y toma de decisiones
    \item Definición temprana de objetivos
    \item Planificación intensificada
    \item Comunicación abierta
    \item Tecnología apropiada
    \item Organización y liderazgo
\end{itemize}

Y las etapas principales son:

\begin{itemize}
    \item Entrenamiento inicial: Metodología de gestión integrada de proyectos
    \item Validación y socialización de los principios y condiciones de satisfacción del proyecto
    \item Estructuración y definición de roles y responsabilidad del equipo participante, además del análisis y elección de metodologías y recursos a utilizar
    \item Aplicación de metodología y herramienta TVD
    \item Sustento legal que deberá ser definido y estructurado para llevar a cabo el proyecto, marcado por la colaboración y la innovación
\end{itemize}

Y los contratos relacionales son:

\begin{itemize}
    \item Ambiente de confianza, comunicación abierta y participación
    \item Cultura corporativa
    \item Trabajo colaborativo y reciprocidad
    \item Relaciones de largo plazo
\end{itemize}

\section{Metodologia Lean Construction}

\begin{figure}[H]
\centering
\includegraphics[width=0.9\textwidth]{IMAGENES/LEAN.png}
\caption{Proceso de Lean Construction}
\label{fig:lean}
\end{figure}

